\documentclass{article}
\usepackage{amsmath}
\usepackage{amsfonts}
\usepackage{amssymb}
\usepackage[margin=0.5in]{geometry}
\usepackage[normalem]{ulem}
\usepackage{graphicx} % Required for inserting images
\usepackage{mathtools}
\usepackage{enumitem}
\usepackage[export]{adjustbox}


\newcommand{\ceil}[1]{\lceil #1 \rceil}
\newcommand{\If}[1]{\textrm{ if #1}}
\newcommand{\then}[1]{\textrm{ then #1}}
\newcommand{\ForAll}[0]{\textrm{ for all }}
\newcommand{\Let}[0]{\textrm{Let }}
\newcommand{\proof}[1]{\textbf{Proof: #1}}
\newcommand{\counter}[1]{\textbf{Counterexample: #1}}
\newcommand{\ns}[1]{\mathbb{#1}}
\newcommand{\Reals}[0]{\ns{R}}
\newcommand{\lub}[1]{\text{lub$_{#1}$}}
\newcommand{\glb}[1]{\text{glb$_{#1}$}}
\newcommand{\pointspace}[3]{\langle#1, #2, #3\rangle}
\newcommand{\pointplane}[2]{\langle#1, #2\rangle}
\newcommand{\inte}[0]{\text{int}}

\newcommand{\set}[1]{\{#1\}}
\newcommand{\osset}{\mathrel{\ooalign{$\subseteq$\cr
  \hidewidth\raise.225ex\hbox{$\circ\mkern.5mu$}\cr}}}
\newcommand{\csset}{\mathrel{\ooalign{$\subseteq$\cr
  \hidewidth\raise.225ex\hbox{$\bullet\mkern.5mu$}\cr}}}
  
\newcommand{\pr}[1]{\operatorname{P}(#1)}

\title{\MakeUppercase{\jobname}}
\author{Justin Nguyen}
\date{\today}

\begin{document}
\maketitle

\begin{enumerate}
  \item $9\cdot 10^6 = 9,000,000$ seven-digit phone numbers. %because 9 choices for the first number, and 10 choices each for the next 6.
  
  \item \begin{enumerate}
    \item $8! = 40320$ seatings %which is all arrangements 8 people.
    \item $5!\cdot 4 \cdot 3! = 2880$ consecutive male seatings. %because there are $5!$ ways the men can sit
    \item $4!\cdot 2^4 = 384$ consecutive couples seatings. %the arrangements of the couples by their swaps
  \end{enumerate}
  
  \item $6m, 7s, 4e \to 2$ books
  \begin{enumerate}
    \item $\binom{6}{2} + \binom{7}{2} + \binom{4}{2} = 42$ choices of the two books who share the same subject. %sum of choosing 2 at a time from each book set
    \item $\binom{6}{1}\binom{7}{1} + \binom{6}{1}\binom{4}{1} + \binom{7}{1}\binom{4}{1} = 94$ choices of two books who don't share the same subject. %sum of products of choosing two books from each set
  \end{enumerate}
  
  \item $8w, 6m \to 3w, 3m$
  \begin{enumerate}
    \item $\binom{8}{3}[\binom{6}{3} - \binom{4}{1}] = 896$ committees where $m_1, m_2$ don't work together.
    \item $\binom{6}{3}[\binom{8}{3} - \binom{6}{1}] = 1000$ committees where $w_1, w_2$ don't work together.
    \item $\binom{8}{3}\binom{6}{3} - P^7_3 = 910$ committees where $w_1, m_1$ don't work together.%P^7_3 = committees where w1,m1 work together
    \footnote{$P^7_3$ is a simplification of the number of committees where $w_1, m_1$ work together.}
  \end{enumerate}
  
  \item Let $D_6 = \set{1,2,3,4,5,6},A =$ rolling $1\ldots 6$ in any order, and $|S| = |D_6|^6$.\\
  If the point $r_1 = (1,2,3,4,5,6) \in A \implies$other points in $A$ must be arrangements of $r_1 \implies |A| = |D_6|!$\\
  $\therefore \pr{A} = \frac{6!}{6^6} = \frac{5}{324} \approx 0.015\ldots$
  
  \item $|S| = \binom{10}{5}$ because the professor chooses 5 questions from 10\\
  $|A| = \binom{6}{5}$ because she chose to study 6 questions, and 5 are on the test\\
  $\therefore \pr{A} = \frac{|A|}{|S|} = \frac{1}{42} \approx 0.02\ldots$
  
  \item $4s_w, 2s_b, 6s_r, 3s_g \to 4s$\\
  $|S| = \binom{15}{4}$ which is all choices of 15 socks taken 4 at a time.
  \begin{enumerate}
    \item We want all possible one color two sock pairs, with another color two sock pairs.\\
    $P(2s_1, 2s_2) = \frac{
      \binom{4}{2}[\binom{2}{2} + \binom{6}{2} + \binom{3}{2}] +
      \binom{2}{2}[\binom{6}{2} + \binom{3}{2}] +
      \binom{6}{2}\binom{3}{2}}
      {\binom{15}{4}} = \frac{177}{1365} \approx 0.13$
    \item At least one red sock is the same as the complement of no red socks.\\
    $P(1s_r) = 1 - P(\text{no reds}) = 1 - \frac{P_4^{9}}{P_4^{15}} = \frac{59}{65} \approx 0.91$
  \end{enumerate}
  
  \item $52$ cards $\to$ 5 cards; $|S| = \binom{52}{5}$ which is how many ways to draw 5 cards from 52 cards.
  \begin{enumerate}
    \item The \#ways to pick 3 aces from 4 by how many ways to pick two kings from 4\\
    $\pr{3A, 2K} = \frac{\binom{4}{3}\binom{4}{2}}{\binom{52}{5}} \approx 0.000009\ldots$
    \item The amount of ways to pick 1 rank from a suite by the amount of ways to pick each rank from a suite.\\
    $\pr{\text{full house}} = \frac{\binom{13}{1}\binom{4}{3}\binom{12}{1}\binom{4}{2}}{\binom{52}{5}} \approx 0.001\ldots$ 
  \end{enumerate}
  
  \item $2w, 4h, 7a \to 1w, 2h, 3a$\\
  If every claim is different, and the process order doesn't matter, then $|S| = \binom{13}{6}$\\
  There is only one way to select $\set{w_1, h_1, h_2, a_1, a_2, a_3} \implies |A| = 1$\\
  $\therefore \pr{A} = \frac{1}{\binom{13}{6}} = \frac{1}{1716} \approx 0.0006\ldots$
  
  \newpage
  \item \begin{enumerate}
    \item Fluke $=\binom{5}{1}\binom{4}{1}\binom{3}{1}\binom{2}{1}\binom{1}{1} = 120$ arrangements
    \item Propose $= \binom{7}{2}\binom{5}{1}\binom{4}{2}\binom{2}{1}\binom{1}{1} = 1260$ arrangements
    \item Mississippi = $\binom{11}{1}\binom{10}{4}\binom{6}{4}\binom{2}{2} = 34650$ arrangements
  \end{enumerate}
  
  \item $3u, 4r, 2z, 1c$; $|S| = \binom{10}{3,4,2,1} = 12600$ rankings\\
  $P(\text{1 winner, 2 losers of US}) = \frac{\binom{7}{2,4,1}\binom{3}{1}\binom{3}{2}}{12600} = \frac{3}{40} = 0.075$
  
  \item $9m = 2m_\alpha + 7m_x \to 3p_1, 3p_2, 3p_3$; $|S| = \binom{9}{3,3,3} = 1680$ outcomes\\
  $P(2m_\alpha \to 3p_1) = \frac{\binom{7}{1,3,3}}{1680} = \frac{1}{12} = 0.08\overline{3}$
  
  \item \proof{}$\sum_{k=0}^{n}\binom{n}{k} = 2^n$\\
  The Binomial Theorem states that $(x+y)^n = \sum_{k=0}^{n}\binom{n}{k}x^{n-k}y^k$.\\
  Then, let $x=y=1 \implies (1+1)^n = (2)^n = \sum_{k=0}^{n}\binom{n}{k}1^{n-k}1^k = \sum_{k=0}^{n}\binom{n}{k}1^n = \sum_{k=0}^{n}\binom{n}{k}$.\\
  $\therefore \sum_{k=0}^{n}\binom{n}{k} = 2^n$.
  
  \item \proof{} $\sum_{k=0}^{n>0}(-1)^k\binom{n}{k} = 0$\\
  If $(x+y)^n = \sum_{k=0}^{n}\binom{n}{k}x^{n-k}y^k \implies (1 - 1)^n = 0^n = \sum_{k=0}^{n}\binom{n}{k}1^{n-k}(-1)^k = \sum_{k=0}^{n}\binom{n}{k}(-1)^k$\\
  $\therefore \sum_{k=0}^{n>0}(-1)^k\binom{n}{k} = 0^n = 0$.
  
  % \item \proof{}$\sum_{k=0}^{n}k\binom{n}{k}=n2^{n-1}$\\
  % Proven earlier, $2^n = \sum_{k=0}^{n}\binom{n}{k} \implies \frac{n}{2}\cdot2^{n} = n2^{n-1} = \frac{n}{2}\sum_{k=0}^{n}\binom{n}{k}$
\end{enumerate}

\end{document}