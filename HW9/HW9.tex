\documentclass{article}
\usepackage{amsmath}
\usepackage{amsthm}
\usepackage{amsfonts}
\usepackage{amssymb}
\usepackage[margin=0.5in]{geometry}
\usepackage[normalem]{ulem}
\usepackage{graphicx} % Required for inserting images
\usepackage{mathtools}
\usepackage[inline]{enumitem}
\usepackage[export]{adjustbox}
\usepackage[super]{nth}
\usepackage[normalem]{ulem}

\newcommand{\ceil}[1]{\lceil #1 \rceil}
\newcommand{\If}[1]{\textrm{ if #1}}
\newcommand{\then}[1]{\textrm{ then #1}}
\newcommand{\ForAll}[0]{\textrm{ for all }}
\newcommand{\Let}[0]{\textrm{Let }}
\newcommand{\counter}[1]{\textbf{Counterexample: #1}}
\newcommand{\ns}[1]{\mathbb{#1}}
\newcommand{\Reals}[0]{\ns{R}}
\newcommand{\lub}[1]{\text{lub$_{#1}$}}
\newcommand{\glb}[1]{\text{glb$_{#1}$}}
\newcommand{\pointspace}[3]{\langle#1, #2, #3\rangle}
\newcommand{\pointplane}[2]{\langle#1, #2\rangle}

\newcommand{\set}[1]{\{#1\}}
\newcommand{\osset}{\mathrel{\ooalign{$\subseteq$\cr
  \hidewidth\raise.225ex\hbox{$\circ\mkern.5mu$}\cr}}}
\newcommand{\csset}{\mathrel{\ooalign{$\subseteq$\cr
  \hidewidth\raise.225ex\hbox{$\bullet\mkern.5mu$}\cr}}}
  
\newcommand{\pr}[1]{\operatorname{P}\left(#1\right)}
\newcommand{\var}[1]{\operatorname{Var}(#1)}
\newcommand{\expt}[1]{\operatorname{E}[#1]}


\newcommand{\binomdist}[3]{#1 \sim \operatorname{Binomial}(#2, #3)}
\newcommand{\geodist}[2]{#1 \sim \operatorname{Geometric}(#2)}
\newcommand{\nbinomdist}[3]{#1 \sim \operatorname{NB}(#2, #3)}
\newcommand{\hypegeodist}[4]{#1 \sim \operatorname{Hypergeometric}(#2, #3, #4)}
\newcommand{\poisson}[2]{#1 \sim \operatorname{Poisson}(#2)}

\newcommand{\unidist}[3]{#1 \sim \operatorname{Uniform}(#2, #3)}
\newcommand{\normdist}[3]{#1 \sim \operatorname{Normal}(#2, #3)}
\newcommand{\expdist}[2]{#1 \sim \operatorname{Exponential}(#2)}
\newcommand{\gamdist}[3]{#1 \sim \operatorname{Gamma}(#2, #3)}
\newcommand{\chidist}[2]{#1 \sim \operatorname{\chi^2}(#2)}
\newcommand{\betdist}[3]{#1 \sim \operatorname{Beta}(#2, #3)}

\newcommand*\diff{\mathop{}\!\mathrm{d}}
\newcommand{\drv}[3]{\frac{\diff#1^{#3}}{\diff#2^{#3}}}
\newcommand{\pdrv}[2]{\frac{\partial#1}{\partial#2}}
\newcommand{\intv}[4]{\int_{#3}^{#4} #1 \diff #2}

\renewcommand{\thefootnote}{\Roman{footnote}}
\newtheorem*{theorem}{Theorem}


\title{\MakeUppercase{\jobname}}
\author{Justin Nguyen}
\date{\today}

\begin{document}
\maketitle

\footnotetext{All decimals are rounded to 4 places.}

\begin{enumerate}
  \item $f(y_1, y_2) = e^{-(y_1 + y_2)}, 0 < y_1, y_2 < \infty 
  \implies f(y_1, y_2) = g(y_1)h(y_2) = e^{-y_1} \cdot e^{-y_2} \implies Y_1, Y_2$ independent.
  \begin{enumerate}
    \item $\pr{Y_1 < 1, Y_2 > 5} = 
    \intv{e^{-y_1}}{y_1}{0}{1} \cdot \intv{e^{-y_2}}{y_2}{5}{\infty}
    = \left[-e^{-y_1}\right]_{0}^{1}\cdot \left[ -e^{-y_2} \right]_{5}^{\infty} 
    = (1-e^{-1})e^{-5}
    \approx 0.0043$
    
    \item $\pr{Y_1 + Y_2 < 3} = $ \[
      \intv{ \intv{e^{-(y_1 + y_2)}}{y_1}{0}{3 - y_2} }{y_2}{0}{3}
      = \intv{ e^{-y_2}\left[ \intv{e^{-y_1}}{y_1}{0}{3 - y_2} \right]}{y_2}{0}{3}
      = \intv{ e^{-y_2}\left[ -e^{-y_1} \right]_{0}^{3-y_2}}{y_2}{0}{3}
      = \intv{ e^{-y_2}- e^{-3}}{y_2}{0}{3}
    \]
    Finally, we arrive at \[
      \left[ -e^{-y_2}- y_2e^{-3} \right]_{0}^{3} = 1 - 4e^{-3} \approx 0.8009
    \]

    \item $\pr{Y_1 < Y_2} = $ \[
      \intv{ \intv{e^{-(y_1 + y_2)}}{y_1}{0}{y_2} }{y_2}{0}{\infty} = 0.5
    \]
    I am too lazy to re-type the work, but it is similar to 1(b).
  \end{enumerate}

  \item $f(y_1, y_2) = y_1^{-1}, 0 < y_2 < y_1 < 1$ \[
    \intv{ \intv{y_1^{-1}}{y_2}{0}{y_1} }{y_1}{0}{1}
    = \intv{ y_1^{-1}\intv{1}{y_2}{0}{y_1} }{y_1}{0}{1}
    = \intv{ y_1^{-1}y_1}{y_1}{0}{1}
    = \intv{ 1}{y_1}{0}{1} = 1
  \]
  $\therefore f(y_1, y_2)$ is a JDF of $Y_1, Y_2$.

  \item $f(y_1, y_2) = y_1 \cdot e^{-y_1y_2} \cdot e^{-y_1}$; $y_1> 0, y_2 > 0$\\
  $f(y_1 \mid y_2) = \frac{f(y_1, y_2)}{f_{y_2}(y_2)}$ 
  % u=y_1, du = 1, v= (y_2+1)^{-1}e^{-y_1(y_2+1)}, dv = e^{-y_1(y_2)+1};
  \[
    f_{y_2}(y_2) 
    = \intv{y_1e^{-y_1(y_2+1)}}{y_1}{0}{\infty}
    = \left[ \frac{-y_1}{y_2+1}e^{-y_1(y_2+1)} + \frac{1}{y_2+1}\intv{e^{-y_1(y_2+1)}}{y_1}{}{} \right]_{0}^{\infty}
    = \left[ e^{-y_1(y_2+1)}
    \left(\frac{1 - y_1y_2 - y_1}{(y_2+1)^2}\right)
    \right]_{0}^{\infty}
  \]
  After evaluating, \[
    f_{y_2}(y_2) = \frac{1}{(y_2+1)^2} \implies f(y_1 \mid y_2) = (y_2+1)^{2}y_1e^{-y_1(y_2+1)}
  \]

  \item $f(y_1, y_2) = c(y_1^2 - y_2^2)e^{-y_1}$, $0 \leq y_1 < \infty$, $-y_1 \leq y_2 \leq y_1$\\
  $f(y_2 \mid y_1) = \frac{f(y_1, y_2)}{f_{y_1}(y_1)}$ 
  \[
    f_{y_1}(y_1) 
    = ce^{-y_1}\intv{(y_1^2 - y_2^2)}{y_2}{-y_1}{y_1}
    = ce^{-y_1}\left[ y_1^2y_2 - \frac{1}{3}y_2^3 \right]_{-y_1}^{y_1}
    = \frac{4c}{3}y_1^3e^{-y_1}
  \]
  So, \[
    f(y_2 \mid y_1) = \frac{3(y_1^2-y_2^2)}{4y_1^3}
  \]
  \newpage

  \item $f(y_1, y_2) = k(1-y_2), 0 \leq y_1 \leq y_2 \leq 1$
  \begin{enumerate}
    \item Solving for k\[
      k\intv{ \intv{1-y_2}{y_1}{0}{y_2} }{y_2}{0}{1}
      = k\intv{ \left[ (1-y_2)y_1 \right]_{0}^{y_2} }{y_2}{0}{1}
      = k\intv{ y_2 - y_2^2 }{y_2}{0}{1}
      = k\left[ \frac{1}{2}y_2^2 - \frac{1}{3}y_2^3 \right]_{0}^{1}
      = \frac{k}{6} \implies k = 6
    \]

    \item $\pr{Y_1 \leq 3/4, Y_2 \geq 1/2} = $ \[
      6\intv{ \intv{1-y_2}{y_1}{0}{1/2} }{y_2}{1/2}{3/4}
      = 6\intv{ \left[ (1-y_2)y_1 \right]_{0}^{1/2} }{y_2}{1/2}{3/4}
      = 6/2\intv{ 1-y_2 }{y_2}{1/2}{3/4}
      = 3 \left[ y_2 - \frac{1}{2}y_2^2 \right]_{1/2}^{3/4}
      = \frac{9}{32}
    \]

    \item \begin{enumerate}
      \item $f_{y_1}(y_1) = $ \[
        6\intv{ 1-y_2 }{y_2}{y_1}{1}
        = 6\left[ y_2-\frac{1}{2}y_2^2 \right]_{y_1}^{1}
        = 3 - 6y_1 + 3y_1^2
      \]
      \item $f_{y_2}(y_2) = $ \[
        6\intv{ 1-y_2 }{y_1}{0}{y_2}
        = 6\intv{ 1-y_2 }{y_1}{0}{y_2}
        = 6y_2(1-y_2)
      \]
    \end{enumerate}

    \item $f(y_1 \mid y_2) = \frac{6(1-y_2)}{6y_2(1-y_2)} = \frac{1}{y_2}$
    \item $f(y_2 \mid y_1) = \frac{6(1-y_2)}{3(y_1-1)^2} = \frac{2(1-y_2)}{(y_1-1)^2}$
    \item $\pr{Y_2 \geq 3/4 \mid Y_1 = 1/2} = $ \[
      \intv{ f(y_2 \mid y_1=1/2) }{y_2}{3/4}{1}
      = 8\intv{ 1-y_2 }{y_2}{3/4}{1}
      = 8\left[ y_2-\frac{1}{2}y_2^2\right]_{3/4}^{1}
      = 1/4
    \]
  \end{enumerate}
  
  \item \begin{enumerate}
    \item $f(y_1, y_2) = y_1e^{-(y_1+y_2)} = y_1e^{-y_1}e^{-y_2} = g(y_1)h(y_2) 
    \implies Y_1, Y_2$ independent.
  
    \item No, $Y_1$ depends on $Y_2$ as part of its domain.
  \end{enumerate}

  \item $f(y_1, y_2) = y_1 + y_2, 0 < y_1, y_2 < 1$ 
  \begin{enumerate}
    \item It is impossible to separate $Y_1, Y_2$'s JDF
    into products of separable single variable functions, therefore it is not independent.
    \item \begin{enumerate}
      \item $f_{y_1}(y_1) = \intv{y_1 +y_2}{y_2}{0}{1} = 
      \left[ y_1y_2 + \frac{1}{2}y_2^2 \right]_{0}^{1} = y_1 + 1/2$
      \item $f_{y_2}(y_2) = \intv{y_1 +y_2}{y_1}{0}{1} = 
      \left[ \frac{1}{2}y_1^2 + y_2y_1  \right]_{0}^{1} = y_2 + 1/2$
    \end{enumerate}

    \item $\intv{ \intv{y_1+y_2}{y_1}{0}{1-y_2} }{y_2}{0}{1} = \frac{1}{3}$
  \end{enumerate}

  \item $\expdist{Y_1}{\lambda_1^{-1}}, \expdist{Y_2}{\lambda_2^{-1}} 
  \implies \pr{Y_1 \leq Y_2} = \frac{\lambda_1}{\lambda_1 + \lambda_2}$
  \begin{proof}
    $f_{Y_1}(y_1) = \lambda_1e^{-\lambda_1y_1}$
    and $f_{Y_2}(y_2) = \lambda_2e^{-\lambda_2y_2} \implies
    f(y_1, y_2) = f_{Y_1}(y_1)f_{Y_2}(y_2)$.\\
    So, $\pr{Y_1 < Y_2}$ \begin{align*}
      % Inner Integral
      = \intv{
        \lambda_2e^{-\lambda_2y_2}
        \intv{ \lambda_1e^{-\lambda_1y_1} }{y_1}{0}{y_2}
      }{y_2}{0}{\infty}
      = \intv{
        \lambda_2e^{-\lambda_2y_2}
        \left[ -e^{-\lambda_1y_1} \right]_{0}^{y_2}
      }{y_2}{0}{\infty}
      = \intv{
        \lambda_2e^{-\lambda_2y_2}
        \left[ 1-e^{-\lambda_1y_2} \right]
      }{y_2}{0}{\infty}\\
      % Outer Integral
      = \lambda_2\intv{
        e^{-\lambda_2y_2}-e^{-y_2(\lambda_2+\lambda_1)}
      }{y_2}{0}{\infty}
      = \left[
        -e^{-\lambda_2y_2} + \frac{\lambda_2}{\lambda_1 + \lambda_2}e^{-y_2(\lambda_1+\lambda_2)}
      \right]_{0}^{\infty}
      = 0 - \left[ -1 + \frac{\lambda_2}{\lambda_1 + \lambda_2} \right]
      = 1 - \frac{\lambda_2}{\lambda_1 + \lambda_2}\\
      = \frac{\lambda_1}{\lambda_1 + \lambda_2}
    \end{align*}
  \end{proof}
\end{enumerate}

\end{document}