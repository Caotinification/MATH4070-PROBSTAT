\documentclass{article}
\usepackage{amsmath}
\usepackage{amsthm}
\usepackage{amsfonts}
\usepackage{amssymb}
\usepackage[margin=0.5in]{geometry}
\usepackage[normalem]{ulem}
\usepackage{graphicx} % Required for inserting images
\usepackage{mathtools}
\usepackage[inline]{enumitem}
\usepackage[export]{adjustbox}
\usepackage[super]{nth}
\usepackage{biblatex}

\newcommand{\ceil}[1]{\lceil #1 \rceil}
\newcommand{\If}[1]{\textrm{ if #1}}
\newcommand{\then}[1]{\textrm{ then #1}}
\newcommand{\ForAll}[0]{\textrm{ for all }}
\newcommand{\Let}[0]{\textrm{Let }}
\newcommand{\counter}[1]{\textbf{Counterexample: #1}}
\newcommand{\ns}[1]{\mathbb{#1}}
\newcommand{\Reals}[0]{\ns{R}}
\newcommand{\lub}[1]{\text{lub$_{#1}$}}
\newcommand{\glb}[1]{\text{glb$_{#1}$}}
\newcommand{\pointspace}[3]{\langle#1, #2, #3\rangle}
\newcommand{\pointplane}[2]{\langle#1, #2\rangle}
\newcommand{\inte}[0]{\text{int}}

\newcommand{\set}[1]{\{#1\}}
\newcommand{\osset}{\mathrel{\ooalign{$\subseteq$\cr
  \hidewidth\raise.225ex\hbox{$\circ\mkern.5mu$}\cr}}}
\newcommand{\csset}{\mathrel{\ooalign{$\subseteq$\cr
  \hidewidth\raise.225ex\hbox{$\bullet\mkern.5mu$}\cr}}}
  
\newcommand{\pr}[1]{\operatorname{P}(#1)}
\newcommand{\var}[1]{\operatorname{Var}(#1)}
\newcommand{\expt}[1]{\operatorname{E}[#1]}


\newcommand{\binomdist}[3]{#1 \sim \operatorname{Binomial}(#2, #3)}
\newcommand{\geodist}[2]{#1 \sim \operatorname{Geometric}(#2)}
\newcommand{\nbinomdist}[3]{#1 \sim \operatorname{NB}(#2, #3)}
\newcommand{\hypegeodist}[4]{#1 \sim \operatorname{Hypergeometric}(#2, #3, #4)}
\newcommand{\poisson}[2]{#1 \sim \operatorname{Poisson}(#2)}


\newcommand*\diff{\mathop{}\!\mathrm{d}}
\newcommand{\drv}[3]{\frac{\diff#1^{#3}}{\diff#2^{#3}}}
\newcommand{\pdrv}[2]{\frac{\partial#1}{\partial#2}}
\newcommand{\intv}[2]{\int #1 \diff #2}

\renewcommand{\thefootnote}{\Roman{footnote}}
\newtheorem*{theorem}{Theorem}


\title{\MakeUppercase{\jobname}}
\author{Justin Nguyen}
\date{\today}

\begin{document}
\maketitle

\footnotetext{All decimals are rounded to 4 places.}

\begin{enumerate}
  \item $\poisson{Y}{3.5} \implies \pr{Y=y} = \frac{3.5^y e^{-3.5}}{y!}$ \begin{enumerate}
    \item $\pr{Y\geq 2} = 1 - \pr{Y=0} - \pr{Y=1} \approx 0.8641$
    \item $\pr{Y \leq 1} = \pr{Y=0} + \pr{Y=1} \approx 0.1359$
  \end{enumerate}
  
  \item $\poisson{Y}{3} \implies \pr{Y=y} = \frac{3^y e^{-3}}{y!}$ \begin{enumerate}
    \item $\pr{Y \geq 3} = 1 - \sum_{y=0}^{2}\pr{Y=y} \approx 0.5768$
    \item \[
      \pr{Y \geq 3 \mid Y \geq 1} 
      = \frac{\pr{Y\geq 3 \cap Y \geq 1}}{\pr{Y\geq 1}}
      = \frac{\pr{Y\geq 3}}{\pr{Y\geq 1}} \approx 0.6070
    \]
  \end{enumerate}
  
  \item $\pr{Y=4} = 3\pr{Y=2}$\\
  Let $\poisson{Y}{\lambda}$ be the number of claims filed $\implies \pr{Y=y} = \frac{\lambda^y e^{-\lambda}}{y!}$.\\
  We solve for $\lambda$, the mean of the number of claims
  \begin{align*}
    \frac{\lambda^4 e^{-\lambda}}{4!} = 3\frac{\lambda^2 e^{-\lambda}}{2!} \implies \lambda = \sqrt{\frac{3\cdot 4!}{2!}} = 6
  \end{align*}
  Then, $\var{Y} = 6$.
  
  \item Let $\poisson{\text{I}}{3}$ and $\poisson{\text{II}}{4}$
  represent the number of cars arriving at entrance \RN{1} and \RN{2} respectively.\\
  The probability that a total of three cars will arrive at the parking lot in a given hour is
  \[
    \pr{\text{I} + \text{II} = 3}
    = \sum_{y=0}^{3}\pr{\text{I} = y}\pr{\text{II} = 3-y} \approx 0.0521
  \]
  Note: $\poisson{Y}{7}$ with $\pr{Y=3}$ is equivalent.
  
  \item Let the number of customers who purchase the item during the day be
  $\poisson{Y}{2} \implies \pr{Y=y} = \frac{2^y e^{-2}}{y!}$\\
  The cost of the item is given by $C(Y) = 100(2^{-Y})$.\\
  The expected cost of the item shall be
  \begin{align*}
    & \expt{C(Y)} = \sum_{y=0}^{\infty}C(y)\pr{Y=y} 
    = \sum_{y=0}^{\infty}100\cdot 2^{-y}\frac{2^y e^{-2}}{y!}\\
    & = 100e^{-2}\sum_{y=0}^{\infty}\frac{1}{y!}\\
    & = 100e^{-2}e = 100e^{-1} \approx \$36.79
  \end{align*}

  \item
  \begin{tabular}{|c|c|c|c|}
    \hline
    $y$ & $\binomdist{Y}{20}{0.05}$ & $\poisson{Y}{1}$ & Difference\\\hline
    0 & 0.35848592 & 0.36787944 & 0.0093935188\\\hline
    1 & 0.3773536 & 0.36787944 & 0.0094741614\\\hline
    2 & 0.1886768 & 0.18393972 & 0.0047370807\\\hline
  \end{tabular}\\
  The binomial distribution approximates the Poisson distribution decently well in this case. The largest error being smaller than 0.01
  
  \newpage
  \item Let $\poisson{Y}{800000 \cdot \binom{30}{6}^{-1}}$ represent the number of
  winners with the approximation for $\lambda = n\cdot p$.\\
  The probability that the state loses money is given by
  \[
    \pr{Y\geq 2} = 1 - \pr{Y=0} - \pr{Y=1} \approx 0.3898
  \]
  which only differs, before rounding, from the binomial distribution's value by $1.0244\times10^{-7}$.

  \item $\pr{Y=y} = \binom{n}{y}p^{y}(1-p)^{n-y}$\\
  Using the Binomial Theorem, it follows that
  \begin{align*}
    & M_Y(t) = \expt{e^{tY}} = \sum_{y=0}^{n}e^{ty}\binom{n}{y}p^{y}(1-p)^{n-y}\\
    & = \sum_{y=0}^{n}\binom{n}{y}(pe^t)^{y}(1-p)^{n-y} = [pe^t + (1-p)]^n
  \end{align*}
  
  \item $M_Y(t) = \frac{1}{6}e^t + \frac{2}{6}e^{2t} + \frac{3}{6}e^{3t}$
  \begin{align*}
    &\expt{Y} = \drv{}{t}{}M_Y(t)\bigg\rvert_{t=0} = 
    \frac{1}{6}e^0 + \frac{4}{6}e^{0} + \frac{9}{6}e^{0} = \frac{14}{6}\\
    &\expt{Y^2} = \drv{}{t}{2}M_Y(t)\bigg\rvert_{t=0} = \frac{1}{6}e^0 + \frac{8}{6}e^{0} + \frac{27}{6}e^{0} = 6\\
    &\var{Y} = \expt{Y^2} - \expt{Y}^2 = 6 - \left( \frac{14}{6} \right)^2 = \frac{5}{9}
  \end{align*}

  \item \begin{enumerate}
    \item \[
      \expt{e^{tY}} = \left[\frac{1}{3}e^t +\frac{2}{3}\right]^5 \implies \binomdist{Y}{5}{1/3}
    \]
    \item \[
      \expt{e^{tY}} = \frac{e^t}{2-e^t} = \frac{pe^t}{1-(1-p)e^t} \implies p = \frac{1}{2} \implies \geodist{Y}{1/2}
    \]
    \item \[
      E[e^{tY}] = e^{2(e^t-1)} \implies \poisson{Y}{2}
    \]
  \end{enumerate}

  \item \begin{enumerate}
    \item $M_Y(0) = \expt{e^0} = \expt{1} = 1$
    \item If $W = 3Y \implies M_W(t) = M_Y(3t)$
    \begin{proof}
      It immediately follows from the definitions
      \[
        M_W(t) = \expt{e^{t3Y}} = M_Y(3t) = \expt{e^{3tY}}
      \]
    \end{proof}
    \item If $U = aY+b \implies M_U(t) = e^{bt}M_Y(at)$
    \begin{proof}
      \[
        M_U(t) = \expt{e^{t(aY+b)}} = \expt{e^{aYt}e^{bt}} = e^{bt}\expt{e^{aYt}} = e^{bt}M_Y(3t)
      \]
    \end{proof}
  \end{enumerate}

  \item If $r(t) = \ln\left[M_Y(t)\right]\implies \drv{}{t}{1}r(t)\bigg\rvert_{t=0} = \expt{Y}$ and $\drv{}{t}{2}r(t)\bigg\rvert_{t=0} = \var{Y}$
  \begin{proof}
    First, we prove that $r^{(1)}(0) = \expt{Y}$
    \begin{align*}
      & M_Y(t) = \expt{e^{tY}} = \sum_{y}e^{ty}\pr{Y=y}\\
      & r(t) = \ln(M_Y(t)) = \ln\left[
        e^{0t}\pr{Y=0} + e^{t}\pr{Y=1} + e^{2t}\pr{Y=2} + \cdots
      \right]\\
      & \drv{}{t}{1}r(t) = 
      \frac{0 \cdot \pr{Y=0} + e^t\pr{Y=1} + 2e^{2t}\pr{Y=2} + \cdots}
      {\pr{Y=0} + e^{t}\pr{Y=1} + e^{2t}\pr{Y=2} + \cdots}\\
      & \drv{}{t}{1}r(t)\bigg\rvert_{t=0} = 
      \frac{0\cdot \pr{Y=0} + 1e^{0}\pr{Y=1} + 2e^{0}\pr{Y=2} + \cdots}
      {\pr{Y=0} + e^{0}\pr{Y=1} + e^{0}\pr{Y=2} + \cdots} = \frac{\expt{Y}}{1} = \expt{Y}
    \end{align*}
    We see that the numerator expands to the definition of $\expt{Y}$,
    and the denominator is the sum of the sample space.\\
    Next, we show the latter, that $r^{(2)}(0) = \var{Y}$
    \begin{align*}
      & \Let f(t) = 0 \cdot \pr{Y=0} + e^t\pr{Y=1} + 2e^{2t}\pr{Y=2} + \cdots 
      \implies \drv{}{t}{}f(t) = 0\cdot P(Y=0) + e^t\pr{Y=1} + 4e^{2t}\pr{Y=2} + \cdots\\
      & \Let g(t) = \pr{Y=0} + e^{t}\pr{Y=1} + e^{2t}\pr{Y=2} + \cdots 
      \implies \drv{}{t}{}g(t) = 0\cdot \pr{Y=0} + e^{t}\pr{Y=1} + 2e^{2t}\pr{Y=2} + \cdots\\
      & \drv{}{t}{2}r(t)\bigg\rvert_{t=0} = 
      \frac{\drv{}{t}{}f(0)g(0) - f(0)\drv{}{t}{}g(0)}{g(0)^2} = \frac{\expt{Y^2}\cdot 1 - \expt{Y}\cdot\expt{Y}}{1} = \expt{Y^2} - \expt{Y}^2 = \var{Y}
    \end{align*}
  \end{proof}
\end{enumerate}

\end{document}