\documentclass{article}
\usepackage[fleqn]{amsmath}
\usepackage{amsfonts}
\usepackage{amssymb}
\usepackage[margin=0.5in]{geometry}
\usepackage[normalem]{ulem}
\usepackage{graphicx} % Required for inserting images
\usepackage{mathtools}
\usepackage[inline]{enumitem}
\usepackage[export]{adjustbox}
\usepackage[super]{nth}


\newcommand{\ceil}[1]{\lceil #1 \rceil}
\newcommand{\If}[1]{\textrm{ if #1}}
\newcommand{\then}[1]{\textrm{ then #1}}
\newcommand{\ForAll}[0]{\textrm{ for all }}
\newcommand{\Let}[0]{\textrm{Let }}
\newcommand{\proof}[1]{\textbf{Proof: #1}}
\newcommand{\counter}[1]{\textbf{Counterexample: #1}}
\newcommand{\ns}[1]{\mathbb{#1}}
\newcommand{\Reals}[0]{\ns{R}}
\newcommand{\lub}[1]{\text{lub$_{#1}$}}
\newcommand{\glb}[1]{\text{glb$_{#1}$}}
\newcommand{\pointspace}[3]{\langle#1, #2, #3\rangle}
\newcommand{\pointplane}[2]{\langle#1, #2\rangle}
\newcommand{\inte}[0]{\text{int}}

\newcommand{\set}[1]{\{#1\}}
\newcommand{\osset}{\mathrel{\ooalign{$\subseteq$\cr
  \hidewidth\raise.225ex\hbox{$\circ\mkern.5mu$}\cr}}}
\newcommand{\csset}{\mathrel{\ooalign{$\subseteq$\cr
  \hidewidth\raise.225ex\hbox{$\bullet\mkern.5mu$}\cr}}}
  
\newcommand{\pr}[1]{\operatorname{P} (#1)}

\renewcommand{\thefootnote}{\Roman{footnote}}

\title{\MakeUppercase{\jobname}}
\author{Justin Nguyen}
\date{\today}

\begin{document}
\maketitle

\begin{enumerate}
  \item $\pr{A} = 0.5, \pr{B} = 0.3, \pr{A\cap B} = 0.1 \implies \pr{A\cup B} = 0.7$
  \begin{enumerate}
    \item $\pr{A \mid B} = \frac{\pr{A\cap B}}{\pr{B}} = \frac{0.1}{0.3} = \frac{1}{3} = 0.\overline{3}$
    \item $\pr{B \mid A} = \frac{\pr{B\cap A}}{\pr{A}} = \frac{0.3}{0.5} = \frac{3}{5} = 0.6$
    \item $\pr{A\mid A\cup B} 
    = \frac{\pr{A \cap (A \cup B)}}{\pr{A \cup B}}
    = \frac{\pr{A} + \pr{A \cup B} - \pr{A \cup (A \cup B)}}{\pr{A \cup B}}
    = \frac{\pr{A} + \pr{A \cup B} - \pr{A \cup B}}{\pr{A \cup B}}
    = \frac{\pr{A}}{\pr{A \cup B}} = \frac{1}{2}$
    \item $\pr{A\mid A\cap B} = \frac{\pr{A \cap (A \cap B)}}{\pr{A \cap B}} = 1$
    \item $\pr{A \cap B\mid A\cup B} 
    = \frac{\pr{(A \cap B) \cap (A\cup B)}}{\pr{A \cup B}}
    = \frac{\pr{((A \cap B) \cap A) \cup  ((A \cap B) \cap B)}}{\pr{A \cup B}}
    = \frac{\pr{A \cap B}}{\pr{A \cup B}} = \frac{1}{7}$
  \end{enumerate}
  
  \item $\pr{A} = 0.2, \pr{B} = 0.3, \pr{A \cup B} = 0.4$ 
  \begin{enumerate}
    \item $\pr{A \cap B} = \pr{A} + \pr{B} - \pr{A \cup B} = 0.1$
    \item $\pr{(A\cap B)^c} = 1 - \pr{A \cap B} = 0.9$
    \item $\pr{(A \cup B)^c} = 1 - \pr{A \cup B} = 0.6$
    \item $\pr{A^c \mid B}
    = \frac{\pr{A^c \cap B}}{\pr{B}}
    = \frac{\pr{B} - \pr{A \cap B}}{\pr{B}} = \frac{2}{3}$
    \footnote{Alternatively, $\pr{A^c \mid B} = 1 - \pr{A \mid B}$}
  \end{enumerate}
  
  \item $\pr{E_1} = 0.9, \pr{E_2 \mid E_1} = 0.8, \pr{E_3 \mid E_1 \cap E_2} = 0.7$\\
  $\pr{E_2} = \pr{E_2 \cap E_1} + \pr{E_2 \cap E_1^c} 
  \implies \pr{E_2} = \pr{E_2 \cap E_1} + \pr{\emptyset}$ (No chance of passing $E_2$ if $E_1$ was failed.)\\
  $\pr{E_2 \mid E_1} = \frac{\pr{E_2 \cap E_1}}{\pr{E_1}} \implies \pr{E_2} = (0.8)(0.9)$\\
  $\pr{E_3 \mid E_2 \cap E_1} = \frac{\pr{E_3 \cap E_2 \cap E_1}}{\pr{E_1 \cap E_2}} \implies \pr{E_3 \cap E_2 \cap E_1} = (0.7)(0.8)(0.9)$\\
  $\therefore \pr{E_3 \cap E_2 \cap E_1} = \frac{63}{125} = 0.504$
  
  \item
  $\pr{6 \mid i} = \begin{cases}
    i \leq 6 \implies 0\\
    i = 7 \implies \frac{1}{6}\\
    i = 8 \implies \frac{1}{5}\\
    i = 9 \implies \frac{1}{4}\\
    i = 10 \implies \frac{1}{3}\\
    i = 11 \implies \frac{1}{2}\\
    i = 12 \implies 1
  \end{cases}$
  
  \item $|S| = \frac{\binom{6}{1}^2}{\binom{2}{1}} = 18, S_5 = \set{\set{1,4}, \set{2,3}}, S_7 = \set{\set{1,6}, \set{2,5}, \set{3,4}}$\\
  $\pr{5} = \frac{1}{9}, \pr{7} = \frac{1}{6}\\
  \pr{5\text{ before }7} = 1\cdot \pr{5} + 0\cdot\pr{7} + (1 - \pr{5} - \pr{7}) = \frac{2}{5}$
  \footnote{  Alternatively, I was shown: $\pr{5\text{ before }7} = \frac{\pr{5}}{\pr{5} + \pr{7}} = \frac{2}{5}$, and this looks like Bayes' Thm.}
  \item \begin{enumerate}
    \item \textbf{Assumptions:} $0 < k \leq n$; 
    $\pr{\nth{1}} = \frac{1}{n}, 
    \pr{\nth{2}} 
    = (1 - \pr{\nth{1}})\cdot \frac{1}{n-1}
    = \frac{n-1}{n}\cdot \frac{1}{n - 1}
    = \frac{1}{n}\\ 
    \implies \pr{k_d^\textsuperscript{th}} = 
    (1 - \pr{\nth{1}})(1 - \pr{\nth{2}})\ldots(1 - \pr{k-2})(1 - \pr{k-1})\frac{1}{n - k + 1} = \frac{1}{n}$\footnote{This is amazing. Why does it work out this way?}
    \item \textbf{Assumptions:} Let $w_k$ denote the wrong $k^{th}$ key.
    $\pr{w_1} = \frac{n-1}{n}, \pr{w_2} = \pr{w_1}^2 \implies \pr{w_k} = \pr{w_1}^k$\\
    The probability of selecting the correct key is $\pr{k_c} = \frac{1}{n}$.\\
    $\therefore \pr{k^{th}} = \pr{w_k}\cdot \pr{k_c} = (\frac{n-1}{n})^k\cdot \frac{1}{n}$
  \end{enumerate}
  
  \item $\pr{R_E \cup R_O} = 85\%, \pr{R_E}= 75\%, \pr{R_E \cap R_O} = \pr{R_E}\pr{R_O}$\\
  $75\%\pr{R_O} = 75\% + P(R_O) - 85\%$\\
  $\therefore \pr{R_O} = 40\%$
  
  \item \begin{enumerate*}[label=(\roman*)]
    \item $\pr{C_C} = 2\pr{C_D}$
    \item $\pr{C_C\cap C_D} = \pr{C_C}\pr{C_D}$
    \item $\pr{C_C\cap C_D} = 0.15$
  \end{enumerate*}\\
  $\pr{C_C}\pr{C_D} = 2\pr{C_D}^2 \implies \pr{C_D} = \frac{\sqrt{30}}{20}, \pr{C_C} = \frac{\sqrt{30}}{10}$\\
  $\pr{(C_C \cup C_D)^c} = 1 - \pr{C_C \cup C_D} = 1 - (\pr{C_C} + \pr{C_D} - \pr{C_C \cap C_D}) = \frac{23-3\sqrt{30}}{20} \approx 0.328\ldots$

  \item $\pr{- | B} = 30\%, \pr{- | A} = 60\%$; $\pr{A} = \frac{1}{4}, \pr{B} = \frac{3}{4}$\\
  $\pr{A \mid -} =
  \frac{\pr{A}\pr{-|A}}{\pr{A}\pr{-|A} + \pr{B}\pr{-|B}} 
  = \frac{2}{5} = 0.4
  $

  \item $\pr{I} = 46\%, \pr{L} = 30\%, \pr{C} = 24\%$\\
  $\pr{V \mid I} = 35\%, \pr{V \mid L} = 62\%, \pr{V \mid C} = 58\%$
  \begin{enumerate}
    \item $\pr{I \mid V} = 
    \frac{\pr{V \mid I}\pr{I}}
    {\pr{V \mid I}\pr{I} + \pr{V \mid L}\pr{L} + \pr{V \mid C}\pr{C}} = \frac{805}{2431} \approx 0.331\ldots$
    
    \item $\pr{L \mid V} = 
    \frac{\pr{V \mid L}\pr{L}}
    {\pr{V \mid I}\pr{I} + \pr{V \mid L}\pr{L} + \pr{V \mid C}\pr{C}} = \frac{930}{2431} \approx 0.382\ldots$
    
    \item $\pr{C \mid V} = 
    \frac{\pr{V \mid C}\pr{C}}
    {\pr{V \mid I}\pr{I} + \pr{V \mid L}\pr{L} + \pr{V \mid C}\pr{C}} = \frac{696}{2431} \approx 0.286\ldots$
    \item $\frac{2431\text{ participating voters}}{5000\text{ voters}} 
    = \frac{(35)(42) + (62)(30) + (58)(24)}{100^2} 
    = \pr{V \mid I}\pr{I} + \pr{V \mid L}\pr{L} + \pr{V \mid C}\pr{C}$\footnote{This is weird. Why does this make sense?}
  \end{enumerate}
  
  \item $B_1 = (b,w), B_2 = (2b,w)$; $\pr{B_1} = \pr{B_2} = \frac{1}{2}$
  \begin{enumerate}
    \item $\pr{b} = \pr{b \mid B_1}\pr{B_1} + \pr{b \mid B_2}\pr{B_2} = \frac{1}{2}\cdot \frac{1}{2} + \frac{1}{2}\cdot \frac{2}{3} = \frac{7}{12}$
    \item $\pr{w} = \frac{5}{12}$\\
    $\pr{B_1 \mid w} = \frac{\pr{w \mid B_1}\pr{B_1}}{\pr{w}} = \frac{3}{5}$
  \end{enumerate}

  \item $D_6 = \set{1\ldots 6}$; $5w + 10b = |D_6|$\\
  $|S_{n\in D_6}| = \binom{15}{n}$ and $\pr{\text{any roll}} = \frac{1}{6}$
  \begin{enumerate}
    \item Let $W_{k\in D_{6}}$ denote the events of getting $k$ white marbles with a $k$ dice roll. Thus, $|W_{k\in D_{6}}| = \binom{5}{k}$. Note $|W_6| = 0$.\\
    Then, $\pr{W} = \sum_{k\in D_6}\frac{|W_k|}{|S_k|} \cdot P(\text{roll }k) = \frac{5}{66} = 0.0\overline{75}$.
    \item $\pr{3} = \frac{1}{6}, \pr{W_3 \mid 3} = \frac{2}{91} \implies \pr{3 \mid W_3} = \frac{\pr{W_3 \mid 3}\pr{3}}{\pr{W_3}} = \frac{1}{6} = 0.1\overline{6}$ 
    \footnote{This makes sense, and is also funny, because the probability of getting a 3 on a dice roll doesn't depend on anything.\\This is akin to proving $x=x$ in a proof.}
  \end{enumerate}
  
  \item Let $A,B: A\cap B = \emptyset$ and $\pr{B} > 0$.\\
  \proof{}$\pr{A \mid A \cup B} = \frac{\pr{A}}{\pr{A} + \pr{B}}$\\
  This immediately follows from $\pr{A \mid A \cup B} 
  = \frac{\pr{A \cap (A \cup B)}}{\pr{A \cup B}} 
  = \frac{\pr{A}}{\pr{A} + \pr{B}}$\\
  $\blacksquare$
  
  \item \proof{}$\pr{A \cap B} \geq 1 - \pr{A^c} - \pr{B^c}$
  \[\pr{A \cap B} = \pr{A} - \pr{A \cap B^c}\]
  \[\pr{A \cap B} = \pr{A} - [\pr{A} + \pr{B^c} - \pr{A \cup B^c}] = \pr{A \cup B^c} - \pr{B^c}\]
  \[\pr{A \cup B^c} - \pr{B^c} \geq 1 - \pr{A^c} - \pr{B^c} \iff \pr{A \cup B^c} \geq 1 - \pr{A^c} \iff \pr{A \cup B^c} \geq \pr{A}\]
  \[\therefore \pr{A \cap B} \geq 1 - \pr{A^c} - \pr{B^c}\]

\end{enumerate}
\end{document}