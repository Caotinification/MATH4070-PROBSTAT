\documentclass{article}
\usepackage{amsmath}
\usepackage{amsthm}
\usepackage{amsfonts}
\usepackage{amssymb}
\usepackage[margin=0.5in]{geometry}
\usepackage[normalem]{ulem}
\usepackage{graphicx} % Required for inserting images
\usepackage{mathtools}
\usepackage[inline]{enumitem}
\usepackage[export]{adjustbox}
\usepackage[super]{nth}
\usepackage[normalem]{ulem}

\newcommand{\ceil}[1]{\lceil #1 \rceil}
\newcommand{\If}[1]{\textrm{ if #1}}
\newcommand{\then}[1]{\textrm{ then #1}}
\newcommand{\ForAll}[0]{\textrm{ for all }}
\newcommand{\Let}[0]{\textrm{Let }}
\newcommand{\counter}[1]{\textbf{Counterexample: #1}}
\newcommand{\ns}[1]{\mathbb{#1}}
\newcommand{\Reals}[0]{\ns{R}}
\newcommand{\lub}[1]{\text{lub$_{#1}$}}
\newcommand{\glb}[1]{\text{glb$_{#1}$}}
\newcommand{\pointspace}[3]{\langle#1, #2, #3\rangle}
\newcommand{\pointplane}[2]{\langle#1, #2\rangle}

\newcommand{\set}[1]{\{#1\}}
\newcommand{\osset}{\mathrel{\ooalign{$\subseteq$\cr
  \hidewidth\raise.225ex\hbox{$\circ\mkern.5mu$}\cr}}}
\newcommand{\csset}{\mathrel{\ooalign{$\subseteq$\cr
  \hidewidth\raise.225ex\hbox{$\bullet\mkern.5mu$}\cr}}}
  
\newcommand{\pr}[1]{\operatorname{P}\left(#1\right)}
\newcommand{\var}[1]{\operatorname{Var}(#1)}
\newcommand{\cov}[1]{\operatorname{Cov}(#1)}
\newcommand{\expt}[1]{\operatorname{E}[#1]}
\newcommand{\iid}{\operatorname{i.i.d.}}

\newcommand{\binomdist}[3]{#1 \sim \operatorname{Binomial}(#2, #3)}
\newcommand{\geodist}[2]{#1 \sim \operatorname{Geometric}(#2)}
\newcommand{\nbinomdist}[3]{#1 \sim \operatorname{NB}(#2, #3)}
\newcommand{\hypegeodist}[4]{#1 \sim \operatorname{Hypergeometric}(#2, #3, #4)}
\newcommand{\poisson}[2]{#1 \sim \operatorname{Poisson}(#2)}

\newcommand{\unidist}[3]{#1 \sim \operatorname{Uniform}(#2, #3)}
\newcommand{\normdist}[3]{#1 \sim \operatorname{Normal}(#2, #3)}
\newcommand{\expdist}[2]{#1 \sim \operatorname{Exponential}(#2)}
\newcommand{\gamdist}[3]{#1 \sim \operatorname{Gamma}(#2, #3)}
\newcommand{\chidist}[2]{#1 \sim \operatorname{\chi^2}(#2)}
\newcommand{\betdist}[3]{#1 \sim \operatorname{Beta}(#2, #3)}

\newcommand*\diff{\mathop{}\!\mathrm{d}}
\newcommand{\drv}[3]{\frac{\diff#1^{#3}}{\diff#2^{#3}}}
\newcommand{\pdrv}[2]{\frac{\partial#1}{\partial#2}}
\newcommand{\intv}[4]{\int_{#3}^{#4} #1 \diff #2}

\renewcommand{\thefootnote}{\Roman{footnote}}
\newtheorem*{theorem}{Theorem}


\title{\MakeUppercase{\jobname}}
\author{Justin Nguyen}
\date{\today}

\begin{document}
\maketitle

\footnotetext{All decimals are rounded to 4 places.}

\begin{enumerate}
  \item $\expdist{\set{Y_1, Y_2, \ldots, Y_6} \iid}{\beta}$\\
  $\pr{\min\set{Y_1, Y_2, \ldots, Y_6} \geq a}
  = \pr{Y_1 \geq \alpha}\pr{Y_2 \geq \alpha}\ldots\pr{Y_6 \geq \alpha} 
  = F_{Y_1}(a)^{6}
  = $ \[
    \left(\intv{\beta^{-1}\exp\left[ -\beta^{-1}y \right]}{y}{a}{\infty}\right)^6
    = \exp\left( -6a/\beta \right)
  \]

  \item $\set{Y_1, Y_2, \ldots Y_n} \iid$
  with pdf $f_Y(y) = e^\theta e^{-y}, y > \theta$\\
  $f_{Y_{(1)}} = n\left[ 1-F_Y(y) \right]^{n-1}\cdot f_Y(y) = $
  \[
    n \left(1 - e^{\theta}\intv{e^{-y}}{y}{\theta}{\infty}\right) e^{\theta}e^{-y}
    = n \left(1 - e^{\theta}\left[ -e^{-y} \right]_{\theta}^{\infty}\right) e^{\theta}e^{-y}
    = n(1 - e^{2\theta} )e^{\theta}e^{-y}
  \]
  
  \item $\betdist{\set{Y_1, \ldots, Y_n} \iid}{2}{2}$ \begin{enumerate}
    \item $F_{Y_{(n)}}(y) = F_Y(y)^n = $ \[
      \left( \frac{\Gamma(2 + 2)}{\Gamma(2)\Gamma(2)}\intv{y^{2-1}(1-y)^{2-1}}{y}{}{} \right)^n
      = \left(\frac{\Gamma(4)}{\Gamma(2)^2}\intv{y(1-y)}{y}{}{}\right)^n
      = \left(3y^2-2y^3\right)^n
      = y^{2n}\sum_{k=0}^{n}\binom{n}{k}\left(3\right)^{k}\left(-2y\right)^{n-k}
    \]
    \item $f_{Y_{(n)}}(y) = \drv{}{y}{}F_{Y_{(n)}}(y) = 6n\left(y-y^2\right)\left(3y^2-2y^3\right)^{n-1}$
  \end{enumerate}

  \item $Y = \unidist{\set{Y_1, Y_2, \ldots ,Y_5}\iid}{0}{1}$\\
  $\pr{ 1/4 < Y_{\phi_{0.5}} < 3/4 } = \pr{1/4 < Y_3 <\frac{3}{4}} = F_{Y(3)}(3/4) - F_{Y(3)}(1/4) = $ \begin{align*}
    &\frac{5!}{(3-1)!(5-3)!}\intv{F_Y(y)^{3-1}\left[ 1-F_Y(y) \right]^{5-3}f_Y(y)}{y}{1/4}{3/4}
    = 30\intv{F_Y(y)^{2}\left[ 1-F_Y(y) \right]^{2}f_Y(y)}{y}{1/4}{3/4}
    \\&= 30\intv{y^2(1-y)^2\cdot 1}{y}{1/4}{3/4}
    = 30\intv{y^2-2y^3+y^4}{y}{1/4}{3/4}
    = 30\left[ \frac{1}{3}y^3 - \frac{2}{4}y^4 + \frac{1}{5}y^5 \right]_{1/4}^{3/4} = \frac{203}{256}
  \end{align*}

  \item \begin{theorem}
    If $\unidist{\set{Y_1, \ldots , Y_n}}{0}{\theta = 1} \implies \betdist{Y_{(k)}}{\alpha}{\beta}$
  \end{theorem}
  \begin{proof}
    We first find the pdf of the $k^{\text{th}}$ order statistic. We know that
    \[
      f_Y(y) = 1 \text{ and } F_Y(y) = y
    \]
    then,
    \[
      f_{Y(k)}(y) = \frac{n!}{(k-1)!(n-k)!}\left[F_Y(y)\right]^{k-1}\left[ 1-F_Y(y) \right]^{n-k}
      = \frac{n!}{(k-1)!(n-k)!}y^{k-1}\left( 1-y \right)^{n-k}
    \]
    A random variable $\betdist{Y}{\alpha}{\beta}$
    iff its pdf is \[
      f_Y(y) = \frac{\Gamma(\alpha + \beta)}{\Gamma(\alpha)\Gamma(\beta)}y^{\alpha - 1}(1-y)^{\beta - 1}
    \]
    Then, we must have that $\alpha = k$ and $\beta = n-k+1 \implies \betdist{Y_{(k)}}{k}{n-k+1}$
  \end{proof}

  \item $\normdist{G_k}{14}{2^2}$
  \begin{enumerate}
    \item Let $G = \set{G_1, \ldots, G_{100}}$.
    We want to know \[
      \pr{\overline{G} > 14.5} = \pr{Z > \frac{14.5 - 14}{2/\sqrt{100}}} = .0062 
    \]
    \item $\pr{g_1 <\overline{G} < g_2} = 0.95$\\
    We have that
    \[
      \pr{g_1 <\overline{G} < g_2} = \pr{\frac{g_1 - 14}{2/\sqrt{100}} < Z < \frac{g_2 - 14}{2/\sqrt{100}}} = 0.95
    \]
    An interval corresponding to $0.95$ means that at either side, we have $2.5\%$ is excluded. Then, \[
      \pr{Z > z_2} = 0.025 \implies z_2 = 1.96 \implies \pr{-1.96 < Z < 1.96} = 0.95
    \]
    So, $\frac{g_1 - 14}{2/\sqrt{100}} = -1.96 \implies g_1 = -1.96\cdot(2/\sqrt{100})+14 = 13.608$, and\\
    $\frac{g_2 - 14}{2/\sqrt{100}} = 1.96 \implies g_2 = 1.96\cdot(2/\sqrt{100})+14 = 14.392$.
  \end{enumerate}

  \item $n = 100, \mu = \mu,\sigma = 2.5 
  \implies H=\normdist{\set{H_k: k\in [1,100]} \iid }{\mu}{2.5^2}$\\
  $\pr{|\overline{H} - \mu| < 0.5} = $ \[
    \pr{\frac{-0.5}{\sigma/\sqrt{n}} < \frac{\overline{H} - \mu}{\sigma/\sqrt{n}} < \frac{0.5}{\sigma/\sqrt{n}}} 
    = \pr{-2 < Z < 2} = 2(\pr{Z > 0} - \pr{Z > 2}) = 0.9544
  \]

  \item $C=\normdist{\set{C_1, \ldots, C_{36}} \iid}{19400}{6000^2}$\\
  $\pr{\overline{C}> 20,000} = $ \[
    \pr{Z > \frac{20000-19400}{6000/\sqrt{36}}} = 0.2743
  \]

  \item $C = \normdist{\set{C_1, \ldots, C_{2025}}}{3125}{540^2}$\\
  $\pr{c_1 < \overline{C} < c_2} = 0.95 = $ \[
    \pr{\frac{c_1 - 3125}{540/\sqrt{2025}} < Z < \frac{c_2 - 3125}{540/\sqrt{2025}}} = 0.95 \implies 
    \frac{c_1 - 3125}{540/\sqrt{2025}} = -1.96 \text{ and }
    \frac{c_2 - 3125}{540/\sqrt{2025}} = 1.96
  \]
  Then, $c_1 = 3148.52, c_2 = 3101.48$.

  \item $\pr{I} = 1/7$; $P = \binomdist{\sum_{k=1}^{612}P_k}{612}{\pr{I}}$\\
  Let $np = 612\pr{I}$, then $\pr{90 < P < 150} = $ \[
    \pr{\frac{90 - np}{\sqrt{np(1-\pr{I})}} < Z < \frac{150 - np}{\sqrt{np(1-\pr{I})}}}
    = \pr{0.2970 - \frac{1}{2} < Z < 7.2281 + \frac{1}{2}} = 0.5804
  \]

  \item $n = 160, p = 0.05$ 
  \begin{enumerate}
    \item $\binomdist{Y}{160}{0.05} 
    \implies \pr{Y \leq 5} = \sum_{y=0}^{5}\binom{160}{y}p^{y}(1-p)^{160-y} = 0.184224295731$
    \item $\poisson{Y}{160 \cdot 0.05} 
    \implies \pr{Y \leq 5} = \sum_{y=0}^{5}\frac{\lambda^{y}e^{-\lambda}}{y!} = 0.19123606208$
    \item $\pr{Y \leq 5} = \pr{Z \leq \frac{5 - np}{\sqrt{np(1-p)}} + \frac{1}{2}} = 0.2810$
  \end{enumerate}
\end{enumerate}

\[
  e = \begin{cases}
    d_{1h}\coloneq D_1(h) + K_1h + K_2h^2\\
    d_{2h}\coloneq D_1(2h) + 2K_1h + 4K_2h^2\\
    d_{3h}\coloneq D_1(3h) + 3K_1h + 9K_2h^2
  \end{cases} 
  \implies e = \alpha d_{1h} + \beta d_{2h} + \gamma d_{3h}
  \implies \begin{cases}
    \alpha + \beta + \gamma = 1\\
    \alpha + 2\beta + 3\gamma = 0\\
    \alpha + 4\beta + 9\gamma = 0
  \end{cases}
\]
$\therefore \alpha = 3, \beta = -3, \gamma = 1$

\end{document}